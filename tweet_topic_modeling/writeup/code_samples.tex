\documentclass[11pt]{article}
\usepackage[normalem]{ulem}
\usepackage[T1]{fontenc}
\usepackage{hyperref}
\usepackage{graphicx}
\hypersetup{
    colorlinks=true,
    linkcolor=blue,
    filecolor=magenta,      
    urlcolor=cyan,
}
% Default margins are too wide all the way around. I reset them here
\setlength{\topmargin}{-.5in}
\setlength{\textheight}{9in}
\setlength{\oddsidemargin}{.125in}
\setlength{\textwidth}{6.25in}
\begin{document}
\title{\includegraphics[scale=0.075]{apple logo grey.png}\\Data Science \& Visualization Specialist -- Product Design\\(ie. the new DataViz teammate's role)}

\author{Candidate:\\Brian Lehman, Viz Farm LLC. \\
\href{tel:+13032465952}{+1 303.246.5952}\\
\href{mailto:brian@viz.farm}{brian@viz.farm}\\ 
\includegraphics[scale=0.2]{viz farm logo black on white.png}\\
}
%\date{\today}
\maketitle 
\tableofcontents

\newpage
\section{DataViz Team Request}
On behalf of the DataViz team, Brad Fuellenbach requested specific work samples. In response to the gracious inquiry of the kind-hearted Brad Fuellenbach, I offer the following details:
\begin{enumerate}
        \item Compulsory sample work (\autoref{sec:authentic}) \textit{\quote A sample of your work demonstrating your statistical data analysis and data visualization skills in R or Python. This should include the actual code and any output produced by the code.\endquote}       
         \item Optional sample work (\autoref{sec:culture} and \autoref{sec:learner}) \textit{\quote Any other sample work you have that you feel would be relevant to this position based on the attached \href{https://jobs.apple.com/en-us/details/200495248}{Job Description} (code, visualizations, presentations, projects, papers, links to websites, etc). Resumes offer only a limited glimpse into a person's capabilities, so use this as an opportunity to tell the team more about you, your passions, and the work that you've done!\endquote}
\end{enumerate}


\section{Candidate Sample Work}
Hello esteemed DataViz team members. As you may have noticed on my \href{https://github.com/blehman/resume/blob/main/Resume-blehman-visual-storyteller.pdf}{resume}, I've held pivotal roles in data science and visualization at leading companies such as Twitter, Google, and PayPal. Confidentiality constraints prevent sharing most work examples from the years on these teams; however, I built a quick project especially for you (\autoref{sec:authentic}) in addition to sharing a few other selections of prior work (\autoref{sec:culture} and \autoref{sec:learner}) . I trust these examples will ignite engaging discussions for us all. 


\subsection{Fluency using statistics and visualizations in Python}\label{sec:authentic}
As a special edition project for the DataViz team at Apple, this work aims to show code examples in a fun way and explore the question: What topics exist in these tweets?
\begin{itemize}
	\item tweet topic modeling: \href{https://github.com/blehman/code\_samples/blob/master/tweet\_topic\_modeling/tom_robbins\_whimsy.ipynb}{notebook} | \href{https://github.com/blehman/code\_samples/tree/master/tweet\_topic\_modeling}{repo}
\end{itemize}. 

\subsection{Passion for creating informative visualizations}\label{sec:culture}
I have a passion for creating informative visualization. This passion has been a mainstay in my life from my early career days at GNIP to building energy consumption visuals at a B-Corp to more recently being interviewed about DS \& viz while working at PayPal. 

\begin{itemize}
	\item Boulder Startup Week presentation on viz development:  \href{https://blehman.github.io/2015-05-13\_BSW\_DataViz\_Lecture/\#/}{presentation link} | \sout{ \href{https://github.com/blehman/2015-05-13_BSW_DataViz_Lecture/tree/master}{repo} }
	\item US energy consumption map \href{https://blehman.github.io/wmo_map/}{viz}\footnote{See an early iteration of the energy consumption map called \href{https://blehman.github.io/tax_story/tax_sounds/}{tax music} for some context on how this project meandered through a many iterative cycles.} |  \href{https://github.com/blehman/wmo_map/tree/gh-pages}{repo} 
	
\end{itemize}


\subsection{Collaborative comrade}\label{sec:learner}
Building environments across teams that foster authentic relating opportunities is a top value of mine. To be able to create feedback loops supporting iterative design, I find it critical that trust is a currency we all trade. As such, collaborative opportunities give my life an enrichment that I carry home after work.  
 \begin{itemize}
 	\item Tye chart collaboration:  \href{https://github.com/twitterdev/Gnip-Trend-Detection}{Kolb's charts} \& \href{https://blehman.github.io/trend\_detection\_graph/}{interactive Tye chart} | \href{https://github.com/blehman/trend\_detection\_graph}{repo}
	\item I was interviewed\footnote{Here's a link to all of the interview details: \href{https://powderkeg.com/the-power-of-data-science-and-visualization-with-brian-lehman-of-honey/}{episode}} on DS \& Viz culture:  \href{https://www.youtube.com/watch?v=uIOiEL5aUe0&t=20m57s}{What makes culture fit?}  | \sout{repo}
 \end{itemize}
\section{Candidate Leading Principles}\label{sec:lead}
The major chords in my life have emerged time and again as a result of being surrounded by brilliant people who know how to hone in on what it takes to develop team dynamics: \newline

 ***********
\begin{itemize}
	\item \textbf{Foster authentic relating.} 
	\item \textbf{Lead as a learner}.
         \item \textbf{Create a culture of iteration}.
\end{itemize}
 
 ***********\newline
The remaining principles were found in pursuit of the first three: \newline

 	***********\
\begin{itemize}
         \item Interpersonally, assume the best.
         \item Share gratitude.
         \item Celebrate success.
         \item Be vulnerable; risk being courageous.
         \item Build camaraderie; risk being vulnerable.
         \item Create opportunities (for everyone).  
         \item Author adventures into the unknown. 
         \item Laugh often.
\end{itemize}

 ***********\newline
My hope is that these principles come to life in the work shared. I look forward to our looming discussions and consider writing this paper an honor. As such, I'm delighted to present you, dear DataViz crew, with a wider lens into my candidacy.

 
\end{document}